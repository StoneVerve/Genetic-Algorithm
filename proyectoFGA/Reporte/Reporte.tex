%Jueves 24, agosto 2017
\documentclass[letterpaper]{article}

\usepackage[utf8]{inputenc}
\usepackage[spanish]{babel}
%\usepackage[spanish]{babel}
%\usepackage{tabularx}
%\usepackage{chronology}
\usepackage{color, colortbl}
\usepackage[table]{xcolor} 
\usepackage{array}

\begin{document}
	\title{Reporte. Proyecto Final FGA}
	\author{Dimitri Semenov Flores\\
			No. Cuenta 313308545 \\
	        Cómputo Evolutivo, Facultad de Ciencias,\\
	        U.N.A.M.}
	\date{16 de enero de 2018}
	\maketitle

	\section{Metodología de comparación y Complicaciones}
	
	\noindent\rule{\textwidth}{1pt}
	
	El objetivo del reporte es presentar los resultados obtenidos
	de forma experimental que comparan el desempeño del algoritmo genético
	simple con el del algoritmo genético fluido.


	Para comparar los algoritmos se pusieron a prueba en seis instancias distintas
	del problema de asignación cuadrático, para cada uno de los algoritmos se
	buscó afinar sus parámetros para cada problema buscando el mejor desempeño
	posible en cada problema.


	No se tomo en cuentra la solución optima definida en la página \textbf{QAPLIB}
	de donde se tomaran las instancias del problema QAP, esto debido a que 
	el costo indicado en la página de la solución optima no concuerda con los 
	valores obtenidos al evaluar la solución con la función objetivo definida.

	Supongo que tal vez en la página se considere algún factor extra en la función
	objetivo que no indican, por lo que se vuelve imposible replicar la función
	objetivo de la página. Por lo anterior decidí comparar los algoritmos 
	en base a quien proporciona la mejor solución.

	\section{Afinamiento de Parámetros}

	\noindent\rule{\textwidth}{1pt}

	Para la afinación del algoritmo genético simple se tomarón tres valores para
	cada uno de los parámetros, los cuales son:
		\begin{itemize}
			\item \textbf{Probabilidad de Cruzamiento}
			
			\item \textbf{Probabilidad de Mutación}
		\end{itemize}
		
	Para cada combinación de los parámetros se realizaron diez experimentos
	y se condidero la mejor solución que generaron.


			
	\begin{center}
		%\rowcolors{2}{blue!25}{blue!5}
		\begin{tabular}[b]{| c | c | c |}
    		\hline
    		\rowcolor{black}
    		\multicolumn{3}{|c|}{\textcolor{white}{\textbf{Chr12a}}} \\
    		\hline
    		\hline
    		Probabilidad Cruza & Probabilidad  Mutación & Mejor solución (costo) \\
    		\hline
    		$0.6$ & $0.1$ & $5050$\\
    		\hline
    		$0.6$ & $0.2$ & $5101$\\
    		\hline
    		$0.6$ & $0.3$ & $5061$\\
    		\hline
    		$0.75$ & $0.1$ & $4776$\\
    		\hline
    		$0.75$ & $0.2$ & $5096$\\
    		\hline
    		$0.75$ & $0.3$ & $4781$\\
    		\hline
    		$0.9$ & $0.1$ & $5123$\\
    		\hline
    		\textbf{0.9} & \textbf{0.2} & \textbf{4776}\\
    		\hline
    		$0.9$ & $0.3$ & $5096$\\
    		\hline
    	\end{tabular}
	\end{center}


	\begin{center}
		%\rowcolors{2}{blue!25}{blue!5}
		\begin{tabular}[b]{| c | c | c |}
    		\hline
    		\rowcolor{black}
    		\multicolumn{3}{|c|}{\textcolor{white}{\textbf{Chr15b}}} \\
    		\hline
    		\hline
    		Probabilidad Cruza & Probabilidad  Mutación & Mejor solución (costo) \\
    		\hline
    		$0.6$ & $0.1$ & $5181$\\
    		\hline
    		$0.6$ & $0.2$ & $4601$\\
    		\hline
    		$0.6$ & $0.3$ & $4641$\\%
    		\hline
    		$0.75$ & $0.1$ & $4754$\\
    		\hline
    		\textbf{0.75} & \textbf{0.2} & \textbf{4320}\\
    		\hline
    		$0.75$ & $0.3$ & $4762$\\
    		\hline
    		$0.9$ & $0.1$ & $5160$\\
    		\hline
    		$0.9$ & $0.2$ & $5394$\\
    		\hline
    		$0.9$ & $0.3$ & $4951$\\
    		\hline
    	\end{tabular}
	\end{center}


	\begin{center}
		%\rowcolors{2}{blue!25}{blue!5}
		\begin{tabular}[b]{| c | c | c |}
    		\hline
    		\rowcolor{black}
    		\multicolumn{3}{|c|}{\textcolor{white}{\textbf{Esc16a}}} \\
    		\hline
    		\hline
    		Probabilidad Cruza & Probabilidad  Mutación & Mejor solución (costo) \\
    		\hline
    		\textbf{0.6} & \textbf{0.1} & \textbf{34}\\
    		\hline
    		$0.6$ & $0.2$ & $36$\\
    		\hline
    		$0.6$ & $0.3$ & $39$\\%
    		\hline
    		$0.75$ & $0.1$ & $35$\\
    		\hline
    		$0.75$ & $0.2$ & $34$\\
    		\hline
    		$0.75$ & $0.3$ & $38$\\
    		\hline
    		$0.9$ & $0.1$ & $34$\\
    		\hline
    		$0.9$ & $0.2$ & $35$\\
    		\hline
    		$0.9$ & $0.3$ & $38$\\
    		\hline
    	\end{tabular}
	\end{center}


	\begin{center}
		%\rowcolors{2}{blue!25}{blue!5}
		\begin{tabular}[b]{| c | c | c |}
    		\hline
    		\rowcolor{black}
    		\multicolumn{3}{|c|}{\textcolor{white}{\textbf{Esc32a}}} \\
    		\hline
    		\hline
    		Probabilidad Cruza & Probabilidad  Mutación & Mejor solución (costo) \\
    		\hline
    		\textbf{0.6} & \textbf{0.1} & \textbf{82}\\
    		\hline
    		$0.6$ & $0.2$ & $114$\\
    		\hline
    		$0.6$ & $0.3$ & $126$\\%
    		\hline
    		$0.75$ & $0.1$ & $87$\\
    		\hline
    		$0.75$ & $0.2$ & $146$\\
    		\hline
    		$0.75$ & $0.3$ & $138$\\
    		\hline
    		$0.9$ & $0.1$ & $83$\\
    		\hline
    		$0.9$ & $0.2$ & $115$\\
    		\hline
    		$0.9$ & $0.3$ & $129$\\
    		\hline
    	\end{tabular}
	\end{center}


	\begin{center}
		%\rowcolors{2}{blue!25}{blue!5}
		\begin{tabular}[b]{| c | c | c |}
    		\hline
    		\rowcolor{black}
    		\multicolumn{3}{|c|}{\textcolor{white}{\textbf{Had20}}} \\
    		\hline
    		\hline
    		Probabilidad Cruza & Probabilidad  Mutación & Mejor solución (costo) \\
    		\hline
    		$0.6$ & $0.1$ & $3737$\\
    		\hline
    		$0.6$ & $0.2$ & $3710$\\
    		\hline
    		$0.6$ & $0.3$ & $3736$\\%
    		\hline
    		$0.75$ & $0.1$ & $3678$\\
    		\hline
    		$0.75$ & $0.2$ & $3728$\\
    		\hline
    		$0.75$ & $0.3$ & $3711$\\
    		\hline
    		\textbf{0.9} & \textbf{0.1} & \textbf{3658}\\
    		\hline
    		$0.9$ & $0.2$ & $3722$\\
    		\hline
    		$0.9$ & $0.3$ & $3691$\\
    		\hline
    	\end{tabular}
	\end{center}

	Para el algoritmo genético fluido se fijo el parámetro \textbf{Diversity rate}
	con el valor de 0.00001 y se escogieron dos valores distintos para cada uno
	de los parámetros restantes, los cuales son:
		\begin{itemize}
			\item \textbf{Probabilidad de Cruzamiento}
			
			\item \textbf{Global Learning Rate}
			
			\item \textbf{Individual Learning Rate}
		\end{itemize}

		\begin{center}
		%\rowcolors{2}{blue!25}{blue!5}
		\begin{tabular}[b]{| c | c | c | c |}
    		\hline
    		\rowcolor{black}
    		\multicolumn{4}{|c|}{\textcolor{white}{\textbf{Chr12a}}} \\
    		\hline
    		\hline
    		Probabilidad Cruza & Global LR &  Individual LR & Mejor solución (costo) \\
    		\hline
    		$0.75$ & $0.1$ & $0.03$ &  $9103$\\
    		\hline
    		$0.75$ & $0.1$ & $0.01$ & $8614$\\
    		\hline
    		\textbf{0.75} & \textbf{0.2} & \textbf{0.03} & \textbf{7370}\\
    		\hline
    		$0.75$ & $0.2$ & $0.01$ & $9119$\\
    		\hline
    		$0.9$ & $0.1$ & $0.03$ &  $8346$\\
    		\hline
    		$0.9$ & $0.1$ & $0.01$ & $8000$\\
    		\hline
    		$0.9$ & $0.2$ & $0.03$ & $8169$\\
    		\hline
    		$0.9$ & $0.2$ & $0.01$ & $8015$\\
    		\hline
    	\end{tabular}
	\end{center}


		\begin{center}
		%\rowcolors{2}{blue!25}{blue!5}
		\begin{tabular}[b]{| c | c | c | c |}
    		\hline
    		\rowcolor{black}
    		\multicolumn{4}{|c|}{\textcolor{white}{\textbf{Chr15b}}} \\
    		\hline
    		\hline
    		Probabilidad Cruza & Global LR &  Individual LR & Mejor solución (costo) \\
    		\hline
    		$0.75$ & $0.1$ & $0.03$ & $11 082$\\
    		\hline
    		$0.75$ & $0.1$ & $0.01$ & $11 866$\\
    		\hline
    		$0.75$ & $0.2$ & $0.03$ & $10 662$\\
    		\hline
    		$0.75$ & $0.2$ & $0.01$ & $12 392$\\
    		\hline
    		$0.9$ & $0.1$ & $0.03$ &  $14 279$\\
    		\hline
    		\textbf{0.9} & \textbf{0.1} & \textbf{0.01} & \textbf{10 601}\\
    		\hline
    		$0.9$ & $0.2$ & $0.03$ & $11 499$\\
    		\hline
    		$0.9$ & $0.2$ & $0.01$ & $12 079$\\
    		\hline
    	\end{tabular}
	\end{center}



		\begin{center}
		%\rowcolors{2}{blue!25}{blue!5}
		\begin{tabular}[b]{| c | c | c | c |}
    		\hline
    		\rowcolor{black}
    		\multicolumn{4}{|c|}{\textcolor{white}{\textbf{Esc16a}}} \\
    		\hline
    		\hline
    		Probabilidad Cruza & Global LR &  Individual LR & Mejor solución (costo) \\
    		\hline
    		$0.75$ & $0.1$ & $0.03$ & $39$\\
    		\hline
    		\textbf{0.75} & \textbf{0.1} & \textbf{0.01} & \textbf{37}\\
    		\hline
    		$0.75$ & $0.2$ & $0.03$ & $40$\\
    		\hline
    		$0.75$ & $0.2$ & $0.01$ & $40$\\
    		\hline
    		$0.9$ & $0.1$ & $0.03$ &  $41$\\
    		\hline
    		$0.9$ & $0.1$ & $0.01$ & $42$\\
    		\hline
    		$0.9$ & $0.2$ & $0.03$ & $40$\\
    		\hline
    		$0.9$ & $0.2$ & $0.01$ & $41$\\
    		\hline
    	\end{tabular}
	\end{center}


		\begin{center}
		%\rowcolors{2}{blue!25}{blue!5}
		\begin{tabular}[b]{| c | c | c | c |}
    		\hline
    		\rowcolor{black}
    		\multicolumn{4}{|c|}{\textcolor{white}{\textbf{Esc32a}}} \\
    		\hline
    		\hline
    		Probabilidad Cruza & Global LR &  Individual LR & Mejor solución (costo) \\
    		\hline
    		$0.75$ & $0.1$ & $0.03$ &  $158$\\
    		\hline
    		$0.75$ & $0.1$ & $0.01$ & $160$\\
    		\hline
    		$0.75$ & $0.2$ & $0.03$ & $151$\\
    		\hline
    		\textbf{0.75} & \textbf{0.2} & \textbf{0.01} & \textbf{150}\\
    		\hline
    		$0.9$ & $0.1$ & $0.03$ &  $164$\\
    		\hline
    		$0.9$ & $0.1$ & $0.01$ & $162$\\
    		\hline
    		$0.9$ & $0.2$ & $0.03$ & $161$\\
    		\hline
    		$0.9$ & $0.2$ & $0.01$ & $164$\\
    		\hline
    	\end{tabular}
	\end{center}

		\begin{center}
		%\rowcolors{2}{blue!25}{blue!5}
		\begin{tabular}[b]{| c | c | c | c |}
    		\hline
    		\rowcolor{black}
    		\multicolumn{4}{|c|}{\textcolor{white}{\textbf{Had20}}} \\
    		\hline
    		\hline
    		Probabilidad Cruza & Global LR &  Individual LR & Mejor solución (costo) \\
    		\hline
    		\textbf{0.75} & \textbf{0.1} & \textbf{0.03} &  \textbf{3651}\\
    		\hline
    		$0.75$ & $0.1$ & $0.01$ & $3657$\\
    		\hline
    		$0.75$ & $0.2$ & $0.03$ & $3704$\\
    		\hline
    		$0.75$ & $0.2$ & $0.01$ & $3713$\\
    		\hline
    		$0.9$ & $0.1$ & $0.03$ &  $3694$\\
    		\hline
    		$0.9$ & $0.1$ & $0.01$ & $3694$\\
    		\hline
    		$0.9$ & $0.2$ & $0.03$ & $3705$\\
    		\hline
    		$0.9$ & $0.2$ & $0.01$ & $3701$\\
    		\hline
    	\end{tabular}
	\end{center}

	\section{AGS vs FGA}
	\noindent\rule{\textwidth}{1pt}

	La siguiente tabla muestra el desempeño de cada algoritmo en cada instancia
	del problema QAP utilizando los parámetros afinados obtenidos para cada
	problema.

	\begin{center}
		%\rowcolors{2}{blue!25}{blue!5}
		\begin{tabular}[b]{| c | c | c |}
    		\hline
    		\rowcolor{black}
    		\multicolumn{3}{|c|}{\textcolor{white}{\textbf{FGA vs AGS}}} \\
    		\hline
    		\hline
    		AGS & FGA & Problema \\
    		\hline
    		$4776$ & $7370$ & chr12a\\
    		\hline
    		$4320$ & $10 601$ & chr15b\\
    		\hline
    		$34$ & $37$ & esc16a\\
    		\hline
    		$82$ & $150$ & esc32a\\
    		\hline
    		$3658$ & $3651$ & had20\\
    		\hline
    	\end{tabular}
	\end{center}

	En base en la tabla anterior encontrarmos que el FGA queda siempre atras
	del algoritmo genético simple en casi todos los problemas y en la mayoría
	es superado con un margen considerablemente alto, esto 
	tal vez se deba al No free lunch theorem.
	
	La única instancia donde el FGA mostro  un mejor desempeño fue
	en la instancia Had20 pero por un margen mínimo.


	\section{Conclusiones}
	\noindent\rule{\textwidth}{1pt}

	En base a los experimentos encontramos que el FGA, a pesar de ser más rapido
	y permitir un control considerablemente más alto sobre la convergencia de
	las soluciones en la población, genera soluciones considerablemente peores o
	marginalmente mejores para instancias del QAP, lo cual creo no es un
	\textit{"trado of"} deseado.

	Una de las principales quejas que tengo con en el fluid genetic algoritm es que 
	en los cromomsomas cuando la probabilidad de predisposición se acerca al valor
	0.0, esa celda del cromosoma simplemente nos dice \textit{Estoy bastante seguro
	de que en esa posición el fenotipo del individuo no debe ser [valor]}.

	Otro problema es que dada la población de cromosomas es complicado obtenter
	el \textit{mejor} de ellos, ya que necesitariamos evaluarlos en base a sus predisposciones
	y es posible que el mejor cromosomas tenga \textit{un mal día} y genere un 
	individuo pésimo.

	Lo cual puede ser exactamente el caso opuesto o para problemas como el 
	Esc32 que tiene 32 valores posibles simplemente nos dice que puede ir alguno
	de los otros 31 valores, pero no nos da alguna pista sobre cual de
	esos valores podria ser el adecuado. Este problema se acentua debido a que
	en base a las pruebas realizadas una gran cantidad de cromosomas tienen 
	varias celdas con predisposiciones cercanas a 0.0, generando varias \textbf{huecos}
	de información sobre la solución optima.


	Quizas si consideramos el teorema de \textit{No free lunch theorem} puede suceder
	que el fluid genetic algorithm se desempeñe mejor que el algoritmo genético 
	simple en otros problemas

	Faltaría realizar un analisis más profundo del funcionamiento y desempeño
	del fluid genetic algorithm, pero al menos con los resultados iniciales decepcionantes
	obtenidos se visualiza un panorama poco emocionante para el fluid genetic algorithm

\end{document}