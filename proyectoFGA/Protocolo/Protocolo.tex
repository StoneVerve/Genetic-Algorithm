\documentclass[12pts,letterpaper]{article}

\usepackage[utf8]{inputenc}
\usepackage[spanish]{babel}
\usepackage{hyperref}
\usepackage{listings}

\begin{document}
	\title{Proyecto Final \\ Quadratic assignment problem}
	\author{Dimitri Semenov Flores 313308545\\
	        Edgar Samuel Perea Domínguez 311117132,\\
	        Computo Evolutivo, Facultad de Ciencias,\\
	        U.N.A.M.}
	\date{10 de diciembre de 2017}
	\maketitle


	\section{Introducción}
		
		El \emph{Quadratic assignment problem} (QAP), en español \emph{El problema
		de la asignación cuadrática} fue planteado por Koopmas y Beckmann en 1957
		como un modelo matemático para un conjunto de actividades económicas indivisibles. 
		Posteriormente Sahni y Gonzales demostraron que el QAP pertenece a los
		problemas no polinomiales duros, lo que sumado a que el problema es
		aplicable a un sinnúmero de situaciones, lo hace un problema de gran
		interés para el estudio.
		
		
		QAP es un problema estándar en la teoría de locación.
		En éste se trata de asignar \textbf{N} instalaciones a una cantidad  \textbf{N} de sitios
		o locaciones en donde se considera un costo asociado a cada una de las
		asignaciones. Este costo dependerá de las distancias y flujo entre las
		instalaciones, además de un costo adicional por instalar cierta instalación
		en cierta locación específica. De este modo se buscará que este costo, en
		función de la distancia y flujo, sea mínimo.

		La dificultad principal de estre problema viene de la naturaleza cuadrática del costo de la función.
		
		
		Con este problema se puede modelar para resolver distintos problemas
		pero para ello se necesitan un gran número de instancias lo cual
		representaría un alto grado de complejidad si se tratara de resolver 
		mediante una técnica completa.
		
		
		Algunas de las aplicaciones de este problema son las siguientes:

		\begin{itemize}
		
			\item \textbf{Centros comerciales}. 
			
				  Diseño de centros comerciales donde se quiere que el público 
				  recorra la menor cantidad de distancia para llegar a tiendas de intereses
				  comunes para un sector del público.

			\item \textbf{Aeropuertos}.
			
				  Diseño de terminales en aeropuertos, en donde se quiere que 
				  los pasajeros que deban hacer un transbordo recorran la 
				  distancia mínima entre una y otra terminal teniendo en cuenta 
				  el flujo de personas entre ellas.

			\item \textbf{Comunicaciones}.
			
				  Procesos de comunicaciones.

			\item \textbf{Teclados}.
			
				  Diseño de teclados de computadora, en donde se quiere por
				  ejemplo ubicar las teclas de una forma tal en que el desplazamientos
				  de los dedos para escribir textos regulares sea el mínimo.

			\item \textbf{Circuitos eléctricos}.
			
				  Diseño de circuitos eléctricos, en donde es de relevante importancia
				  dónde se ubican ciertas partes o chips con el fin de minimizar 
				  la distancia entre ellos, ya que las conexiones son de alto costo.
		
		\end{itemize}

	\section{Hipótesis}

		Basandonos en el artículo expuesto en clase sobre el \emph{fluid genetic algorithm}
		(FGA) buscarames realizar una implementación de FGA que resuelva de forma 
		exitosa una instancia del problema QAP y también realizaremos un 
		algoritmo genético simple que busque resolver las mismas instancias del 
		QAP.

		Dado lo que el artículo nos presenta sobre el FGA,
		lo que esperamos es ver mejores resultados al usar este algoritmo en comparación
		con el algoritmo genético simple. El algoritmo que usaremos como ya se mencionó,
		sera el algoritmo genetico fluido.
		
		El FGA difiere del simple especialmente 
		en la codificacion de los individuos, pues estos son diferentes a los cromosomas,
		los cuales son una especie de predispocicion para los individuos que son creados
		a partir de este con una funcion en la que tambien se difiere del algoritmo simple,
		la funcion \emph{Born-an-Individual}. Dado que la representación de los individuos es
		diferente también, el crossover y la mutacion cambian, el crossover cruza no
		solo al indiviuo, si no que también cruza al cromosoma del que el individuo 
		fue generado junto a este, y el crossover se elimina argumentando que no es
		necesario pues con la función Born-an-Individual se maneja muy bien el 
		problema de las convergencias prematuras.
		


	\section{Metodología}

		Volviendo al artículo expuesto del FGA y utilizando la pagina 
		\emph{QAPLIB - A Quadratic Assignment Problem Library} tomaremos las mismas
		instancias de problemas del tipo QAP que fueron usadas en al artículo para
		corroborar los resultados que se presentaron en el artículo, además se tomara
		otra instancia que consideramos de interés
		
		Las instancias usadas en el artículo son: Chr12a,Chr15b,Had10,Esc16a y 
		Esc32a\\
		Además se realizara el extra Kra30a
		
		Dado lo anterior mediremos el desempeño de los algoritmos verificando que
		tanto se aproximan las soluciones que dan a la soluciones óptimas 
		proporcionadas en QAPLIB (Como en el artíctulo de FGA).
		
		Buscaremos realizar 10 experimentos por problema para cada uno de los
		dos algoritmos, de donde tomaremos la mejor solución dada por cada uno
		para evaluar su desempeño.
		
		También se buscara tratar de afinar los parámetros correspondientes para
		cada uno de los algoritmos y para cada problema de forma experimental, donde
		tomaremos 3 valores distintos para cada uno de los parámetros de entrada,
		obteniendo $3^{3}$ posibles combinaciones para el AGS y $3^{4}$ para el FGA.
		

	\begin{thebibliography}{6}
		\bibitem{descripcion_QAP}
			\emph{Revision de QAP}\\
			Tomado el 9 de diciembre de 2017.\\
			http://www.sciencedirect.com/science/article/pii/S0377221705008337
		\bibitem{wikipedia_modelo_jerarquico}
			\emph{Planteamiento del QAP}\\
			Tomado el 9 de diciembre de 2017.\\
			https://neos-guide.org/content/quadratic-assignment-problem
		\bibitem{wikipedia_modelo_red}
			\emph{QAP}\\
			Tomado el 9 de diciembre de 2017.\\
			http://www.localsolver.com/documentation/exampletour/qap.html
		\bibitem{wikipedia_modelo_objetos}
			\emph{QAPLIB - A Quadratic Assignment Problem Library}\\
			Tomado el 9 de diciembre de 2017.\\
			http://anjos.mgi.polymtl.ca/qaplib/

	\end{thebibliography}

\end{document}

